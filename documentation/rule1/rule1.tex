\documentclass[a4paper]{jpconf}
\usepackage{graphicx}

\newcommand{\ruleone}{Rule \#1\space} 

\begin{document}

\title{\ruleone Investing}

\author{}

\address{}

\ead{robert.bainbridge@gmail.com}

%\begin{abstract}
%\ruleone abstract.
%\end{abstract}

\section{About the Author}

An ex-Green Beret and former river guide, Phil Town is a self-made
millionaire several times over and America's most widely sought-after
speaker on investing. 

In his book, \ruleone (March 2006; paperback, August 2007), he
describes the \ruleone personal financial strategy in detail so that
anyone, even first-time investors, can get and stay rich.  Phil Town
is the classic Everyman, albeit one whose education and resources were
more limited than most.

An average high school student, he completed college on his fourth
try. Of his early working years, he says he "mostly got dirty for a
living", taking on jobs such as digging ditches and pumping gas. Town
spent three and a half years in the Army, returned from the Vietnam
War and found a job in the States as a river guide.  

Drifting through California, Utah, and Idaho, he subsisted at poverty
level, combining his wages from the guiding season and
unemployment. He wore black leathers, sported a goatee, lived in a
teepee in the woods near Flagstaff, Arizona, and "drove around in a
really loud black Harley Davidson".

Phil Town appears regularly on the same dais as Rudy Giuliani, Jimmy
Carter, and Colin Powell as part of the Get Motivated touring success
seminar. He speaks to more than 500,000 people annually about
\ruleone. He is a regular guest on CNBC's "The Millionaire Inside",
alongside fellow wealth experts like David Bach, Barbara Corcoran,
Loral Langemeier, and frequently appears on the MSNBC program "Your
Business".

To know more about the author, go to: http://www.philtown.typepad.com/

\section{The Big Idea}

Most Americans are trapped in mutual funds that, at best, ride the
waves of the market. They diversify to spread the risk. They are in
this sort of investment for the long haul. But they still lose money
in market downturns.

However, the confluence of technology, money and strategy is creating
a revolution in investing at a time when small investors need it the
most. Thanks to the Internet, people know a great deal more than they
used to and can access this information very quickly, plus they can
move in and out of markets far faster than ever before. 

The book is a simple guide to returns of 15 percent or more in the
stock market, with almost no risk. As a matter of fact, \ruleone
investing is practically immune to the ups and downs of the stock
market. \ruleone is the result of the one tried-and- true investing
strategy meeting institutional control of the market at a time when
the tools of investing are available to anyone with a computer. 

The little guy who doesn't have eight hours a day to conduct
exhaustive market research can implement \ruleone - the tools to do so
are already on everyone's computers! 

\section{The Myths of Investing}

To start, let's dissect some prevailing myths regarding investing
today.

\begin{itemize}
\item {\bf You have to be an expert to manage money.} This would be
  true if investing were hard to learn, or getting the information to
  make a decision took a lot of time. But neither is the case any more
  these days, thanks to the Internet. The tools that used to cost
  \$50,000 a year can be gotten for less than \$2 a day and take only
  minutes a day to use - and are even more accurate than what your
  fund manager used a year back.

\item {\bf You can't beat the market.} Efficient Market Theory (EMT)
  holds that markets in general, and the stock market in particular,
  are efficient - they price things according to their value, such
  that the price of the stock at all times equals the price of the
  value of the company. However, some people really do beat the market
  for long periods (as will be shown in the succeeding section).

\item {\bf The best way to minimize risk is to diversify and hold (for
    the long term).} Not necessarily true - if you know how to invest
  - meaning you understand \ruleone and know how to find a wonderful
  company at a very attractive price - you will not need to diversify 
  your money into 50 stocks or an index mutual fund. You'll focus on a
  few businesses you understand, buy when the fund managers are
  fearful, and sell when they're greedy.  4. Dollar cost averaging
  (DCA) will protect you. It won't. DCA is the strategy of buying
  stocks or mutual funds every month with the same amount of money,
  regardless of the price of the stock or fund. Its objective is to
  minimize your investment risk by making the average cost per share
  of stock smaller. However, in a long sideways or down market, it's
  the same as buy-and-hold, and you have to put in the same amount
  every month no matter what.

\item {\bf Real estate is a better investment than businesses.}  Not
  always true either. A reasonably good growth rate in a real estate
  market over a 30- year period is about 4 percent. A reasonably good
  rate of return for a \ruleone investor is 15 percent.

\end{itemize}

\section{\ruleone and The Four Ms}

\subsection{\ruleone in a nutshell}

Knowing you will make money - certainty - comes from buying a
wonderful business at an attractive price. In essence, it's just about
being a good shopper. 

\ruleone investing, then, comes down to four straightforward steps:
\begin{itemize}
\item Find a wonderful business.
\item Know what it is worth as a business.
\item Buy it at 50\% off.
\item Repeat until very rich.
\end{itemize}

\subsection{The Four Ms}

Use these as questions to evaluate a business and decide whether it
can be a good investment: 

{\bf 1. Does the business have meaning to you?}

This implies two other questions:
\begin{itemize}
\item Do you want to own the whole business?
\item Do you understand it well enough to own all of it?
\end{itemize}

You will want to think like a business owner and not just a stock
investor; this is critical to becoming a successful \ruleone
investor. If you buy the business as a business and not just a stock
speculation, then it becomes personal and you want to be proud of what
you own. 

Buy every business with the 10-10 Rule in mind: I won't own this
business for ten minutes unless I'm willing to own it for ten
years. As a \ruleone investor you will own only a few businesses, so
you must prepare to be certain you own the right few businesses that
won't lose you money. 

What's important here is to {\bf understand your businesses inside and
  out}. Start your search for a wonderful business by discovering the
kinds of business you already understand. To do this, ask yourself
these questions:

\begin{itemize}
\item What do you love to do, professionally and as recreation?
\item What things are you really good at?
\item What do you do to make money, or spend your money on?
\end{itemize}

\end{document}


